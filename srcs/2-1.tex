\subsection{动态环境下静态部分地图的构建} motion removal
\label{subsec:object-centered_mapping}
由于便携的消费级深度传感器的出现,室内场景的稠密视觉SLAM在近些年取得了不错的进展。KinectFusion~\cite{kinectFusion}首次利用RGBD数据实现了实时的稠密定位和数据融合,并在场景尺度~\cite{voxelHashing}、回环调整~\cite{kintinuous}以及计算效率~\cite{CPUmapping}上有着一系列的拓展。这类方法建立在场景完全静态的严格假设下。而对于存在运动物体的场景,动态的观测数据需要被视为离群点从地图中剔除,以避免被定位模块使用。

ElasticFusion~\cite{elasticFusion}可以应对画面中存在少量运动物体的场景。算法并未显式地检测运动物体,而是将动态环境下的稠密重建作为一个鲁棒估计问题,通过统计的方式自主地将动态区域作为外点剔除。在这个工作的基础上,\cite{keller13_3dv} 从重建的角度出发,认为每个面元只有在多个连续帧被反复观测到才可以融合到三维模型中。当输入的点云数据与匹配上的地图点位置距离过远时,这部分点云会被作为种子点,通过区域生长将当前帧分割成静态和动态区域。相应地,地图上与动态区域有着匹配关系的部分将从地图上剔除掉。

BaMVO~\cite{BaMVO} 利用背景提取领域(background subtraction)广泛使用的非参数化背景模型进行稠密视觉里程计估计(dense visual odometry)。通过存储连续的4帧深度图并对齐到同一个视角,背景区域可以根据多帧对齐后的深度值差异来进行判别。这样的多帧判别方法建立了时域上的连续性,但是由于采用帧到帧(frame-to-frame)的定位策略,BaMVO不可避免地引入了累计误差。

BaMVO说明时序多帧的反馈对动态环境下有效的运动物体检测与分割至关重要。StaticFusion~\cite{staticFusion}证明维护一个只包含场景中静态部分的三维地图是一种有效地时序信息传播的方式。通过三维数据融合,这种长时的时序信息不会带来额外的计算代价。通过同时检测运动物体并重建静态环境,staticFusion实现了动态环境下的鲁棒稠密的RGBD SLAM。由于采用了帧到模型(frame-to-model)的定位策略,相机位姿估计可以有效地消除由于累计误差带来的漂移。



